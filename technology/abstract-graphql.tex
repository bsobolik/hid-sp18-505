\section{GraphQL}

{\bf please replace urls with citations}

{\bf please use  hid as prefix to bibtex labels}

Data services have been an important component in the evolution of the
information age.  In the early 2000s data-based web services relied
heavily on structured data formats like Extensible Markup
Language (XML).  Other data formats also included raw or plain text, or
serialized data objects.  Whether for public or private use, in most
cases there was and has been a necessity for documenting data
services.  In other words, web service consumers need to know what
data is available and how to query that data.  In the case of XML,
web-services were typically developed with \href{https://en.wikipedia.org/wiki/Web_Services_Description_Language}
{Web Services Description Language} (WSDL) as part of the service.

Over time, challenging issues related to using XML as a data delivery format 
emerged.  The \href{https://en.wikipedia.org/wiki/JSON}{JSON} data format in 
conjunction with the \href{https://en.wikipedia.org/wiki/Representational_state_transfer}{REST} 
architecture emerged as an alternative to XML and SOAP.  However, REST also has 
challenges in terms of documenting the services and data available in this 
type of REST architecture.

``GraphQL has emerged as query language that can reside on top of the REST 
architecture and address many of the issues associated with using XML/SOAP and 
JSON/REST''~\cite{GraphQL2018}.
